\mysubsection{Sarah Häfele}{Problemanalyse}

Dieses abschließende Kapitel soll kurz alle wichtigen Designentscheidungs- und technischen Probleme zusammenfassen und einen Ausblick geben. 

\begin{itemize}
\item Gravitation: Die Gravitation wird nur durch Skripte simuliert, da sonst jede Würfelseite eine eigene Gravitationskraft haben müsste.
\item Perspektive und Steuerung: Die Position des Spielers in der Mitte des Würfels wirft einige Fragen auf, wie mit der Welt interagiert werden soll - durch Reisen auf die einzelnen Seiten oder durch freies Fliegen durch den Raum. Letzteres würde besser zum Thema passen, ist aber zu frei für ein Rätselspiel.
\item Mobiles Endgerät nicht leistungsstark
\item Handykamera Dekay von einer Sekunde - möglicherweise durch das Android Plugin
\item Intro ruckelt
\item Modelle sind aus zu vielen Einzelteilen modelliert
\item Modelle werden zum Teil wegen falschen Normalen nicht richtig angezeigt - hierfür wurde ein Shader umgeschrieben, so dass das culling ausgeschalten ist.
\item Texturen UV Mapping muss zum Teil per Hand koordiniert werden
\item Die Koordinatensysteme der Modelle und der Welt passen manchmal nicht zusammen, was zu Problemen mit Skripten führt
\end{itemize}

Der Prototyp hat zu wenig Spielerführung und benötigt mehr Level mit Savestates. Eine weitere interessante Aufgabe wäre es, die Leap Motion oder ähnliches als Eingabegerät einzubauen.