\subsection{Designentscheidungen}
\mysubsubsection{Lydia Friedrich}{Relationen und Größen der Modelle}
 
Ein gravierendes Problem welches bei der Gestaltung der einzelnen Welten auftrat war, die zuvor nicht einschätzbaren Relationen der Objekte in Cinema4D und der tatsächlichen Größe der Landschaft in Unity. Zumahl sich die Größe der Lanschaften im Entwicklungsprozess einige male veränderte. Zieht man zum Beispiel ein Objekt mit einer Höhe von ca. 2m in eine Welt mit den Maßen 200x200m hineingezogen, wirkt das Model innerhalb der Welt auf einmal winzig. Durch Maßstabsgetreues Einsetzen der Objekte wirkt die Landschaft sehr leer. 

Daraus ergaben sich mehrere Probleme: ursprünglich war gedacht die Welt mit vielen Details und vielen Objekten auszustatten. Auf Grund der auszufüllenden Fläche, würden jedoch enorm viele Objekte benötigt werden um den Eindruck einer belebten Landschaft zu erzeugen. Weiterhin würde die enorm hohe Anzahl der Modelle, eine ebenfalls hohe Anzahl an Polygonen erzeugen, welche wiederum die Lauffähigkeit des Spiels beeinträchtigen könnten. Aus diesen Gründen wurden die umfangreichen Details für die Welten nicht realisiert. Stattdessen lag die Fokussierung der detailreichen Modelle in den aktiven Zonen der einzelnen Welten, d.h.:nur in diesen Bereichen wurden Details mehrfach in Originalgröße eingesetzt. Um die restliche Landschaft belebbar und anschaulich zu gestalten wurden die Objekte über proportioniert dargestellt. Ein weiter Punkt der zu dieser Entscheidung geführt hat war, dass die Ansicht der Objekte aus weiten Entfernungen so winzig waren, dass man nicht erkennen konnte, was diese Objekte eigentlich darstellen.

\mysubsubsection{Sandra Beuck}{Charaktere}