\mysubsection{Sarah Häfele}{Feature Set}

InCubed wird auf einem mobilen Gerät mit der ZEISS-VR ONE gespielt, was besondere Features ermöglicht. Das Spiel ist dadurch nicht nur mobil, sondern wird zu einer VR- und AR-Anwendung:

Die stereoskopischen Linsen der Brille geben dem Nutzer einen Tiefeneindruck in die Welt. Durch das Headtracking werden die Kopfbewegungen des Spielers in die Spielwelt übertragen. So wird das Eintauchen in die Geschichte unterstützt.

Die Brille hat im Gegensatz zu anderen VR-Headsets zusätzlich ein transparentes Sichtfeld, welches in der Anwendung ebenfalls zum Einsatz komm. Der Spieler steuert die Welt durch einen Markerwürfel, den er physikalisch in der Hand hält. 

Dieses Zusammenspiel zwischen Realität und Virtualität wird noch durch die Geschichte aufgegriffen.

Zudem kommt der Spieler durch den Marker und durch das Headtracking komplett ohne weitere Steuerung aus. Er bewegt sich durch seine Blickrichtung in der Welt fort.