\mysubsection{Sarah Häfele}{Story und Setting}

Das Setting ist eine kleine Insel in Mitten von Wasser. Ein verrückter Wissenschaftler entdeckt beim ersten Probeflug seines neu erfundenen Flugapparats, dass die Insel ein aufgeklappter Würfel ist. Seine Freunde und alle Bewohner der Insel glauben ihm nicht und lachen ihn aus. Gedemütigt baut der Wissenschaftler eine Maschine, die mit Hilfe von Gravitation die Insel zusammenklappt, um allen seine Theorie zu beweisen. Durch die Wucht der aufeinander treffenden Seiten kommt ein großer Sturm auf, der den Remote Cube (die Fernbedienung, die die Welt wieder auseinander klappen lässt) aus seiner Hand reißt und die Einzelteile in alle Winde zerstreut.

Die kleine Insel, jetzt als Würfel zusammengeklappt, stürzt ins Chaos und verdunkelt sich. Dem Wissenschaftler gelingt es jedoch kurz vorher eine Flaschenpost mit einem Hilferuf in die Welt zu schicken. Der Spieler findet diese Flaschenpost (ein Würfel, wie kann es denn anders sein) und wird mit ihm auf die Insel teleportiert. Die Flaschenpost ist gleichzeitig der Marker, den der Spieler physikalisch in der Hand hält. 

In der Dunkelheit angekommen, wird er vom Wissenschaftler gebeten, die kleine Insel zu retten. Dafür muss er alle Teile des zerstörten Remote Cubes wiederfinden. Die einzige andere Hilfestellung des Wissenschaftlers sind die leicht kaputten Gravitationsmaschinen, die auf jeder Seite des Würfels verteilt sind.

Hat der Spieler alle Teile des Remote Cubes gefunden und wieder zusammengesetzt, kann die Insel wieder auseinander geklappt werden. Der Spieler sieht, dass der Wissenschaftler recht hatte, denn die Insel zeigt einen aufgeklappten Würfel. Der Spieler wird durch den Zeitreisetunnel wieder in die reale Welt geschickt, die zum Schluss wieder eingeblendet wird.