\subsection{Intro und Outro}
\mysubsubsection{Fabian Gärtner}{Technik}

BILDER ZU INTRO/OUTRO

Das Spiel soll von Anfang an den Spieler zur Hauptrolle machen, weshalb das Intro und Outro so gestaltet wurde, dass es dem Spieler deutlich vermittelt, dass er persönlich vom verrücken Wissenschaflter zu Hilfe gerufen wurde. Wenn der Spieler das Spiel startet und die Brille aufsetzt, sieht daher er zunächst - nach einem Ladebildschirm - über die Kamera des Smartphones die Realwelt vor sich. Ein Text auf dem Bildschirm erklärt anschließend, dass der Spieler nun seinen magischen Würfel wie gezeigt (mit dem Stern zur Kamera) in das Bild halten soll. Tut er dies für fünf Sekunden beginnt eine Sequenz, die den Spieler aus der Realwelt in die Spielwelt teleportiert. Auch hier wird das Spiel der Themenvorgabe gerecht, da der Spieler eine Reise von der Realität in die virtuelle Welt, also zwischen verschiedenen Dimensionen, miterlebt. Vermittelt wird dem Spieler diese Reise visuell durch das Einfrieren, Drehen, Verschwimmen und Überlichten des Kamerabildes und das anschließende Einblenden einer drehenden Spirale. Gleichzeitig wird diese Sequenz als eine Einleitung in das Spiel genutzt, indem dem Spieler die Vorgeschichte visuell über sechs hereinfliegende Comics und auditiv durch einen Erzähler vermittelt wird. Nach diesem Intro landet der Spieler innerhalb des dunklen Würfels und kann mit der Erkundung der Welt beginnen.


\mysubsubsection{Sandra Beuck}{Art}

Hallo