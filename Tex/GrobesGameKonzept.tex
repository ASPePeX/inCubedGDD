\mysubsection{Sarah Häfele}{Grobes Game Konzept}

Die sechs Innenseiten eines großen Würfels stellen Spielwelten dar. Die Objekt-Models sind im Lowpoly-Style gehalten, da sie so schnell modelliert werden können (Zeit ist im Game Jam ein wichtiger Faktor), das Mobile Device diese gut darstellen kann und zudem das eckige Würfelthema so wieder aufgegriffen wird.

Der Spieler befindet sich im Zentrum des Inneren eines Würfels und sieht die Welten im First-Person-View. Jede Innenseite besitzt eine unterschiedliche Welt, wobei jede Welt im Bezug auf ihre gegenüberliegende steht und mit ihrem Paar zwei gegensätzliche Themen darstellen.    

Die Themen sind:
\begin{itemize}
\item Feuer vs. Eis
\item Wald vs. Wüste
\item Wiese vs. Gebirge
\end{itemize}

Pro Paar müssen Aufgaben durch eine Abfolge verschiedener Aktionen in beider sich gegenüberstehender Welten gelöst werden. Eventuell können verschiedene Items als Hilfe benutzt werden. Hauptziel ist es, die Teile eines kleinen Würfels wiederzufinden.

Der Spieler hält einen Markerwürfel in der Hand, der mir verschiedenen Farbflächen und Mustern markiert ist. Mit diesem kann er durch Markererkennung die virtuelle Welt drehen. Damit wird auch in der \enquote{realen} Welt wieder das Würfelthema aufgegriffen.

Der Spieler bewegt sich rein durch seinen Blick durch die Welt. Er startert immer in der Mitte des Würfels. Jede Seite hat einen Viewpoint, auf den der Spieler, wenn er ihn länger anschaut, zugleitet. Damit kann der Spieler in die Welt eintauchen und die Seiten erkunden. Wieder zurück in der Mitte kann der Spieler sich Orientierung verschaffen und die Auswirkungen seiner Aktionen überblicken.

Das Spiel heißt \enquote{inCubed}, da das englische Wort für \enquote{Kubik} oder \enquote{hoch drei} \enquote{cubed} ist und der Spieler sich in einem dreidimensionalen Würfel befindet.

Das Game Jam Thema wird durch die gleichberechtigten Seiten aufgegriffen: wenn man auf einer Seite etwas verändert, ändert sich auch etwas auf der anderen. Durch das Umschalten der Gravitation, kann man sich frei in alle Dimensionen bewegen. In einem fertigen Spiel sollten alle Seiten voneinander gleichberechtigt abhängig voneinander sein. Für den Prototypen sollen nur die sich gegenüberliegenden Wände voneinander abhängen.    