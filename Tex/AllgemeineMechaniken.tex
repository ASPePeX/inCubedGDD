\mysubsection{Sarah Häfele}{Allgemeine Struktur}

Die zunächst auffälligste Mechanik zu Beginn des Spieles ist die Dunkelheit, die durch die erste Aufgabe überwunden wird. Der Dunkel-Hell-Zyklus soll dem Spieler nicht nur Führung in der ersten Aufgabe bieten, sondern auch für spätere Level ein wichtiger Mechanismus werden. So kann der Spieler die Welt dunkel machen, um leuchtende Gegenstände, mit denen er interagieren kann, besser sehen zu können. Im Prototypen hat der Dunkel-Hell-Zyklus nur im Tutorial Bedeutung.

Die Gravitation ist ein weiterer wichtiger Spielmechanismus. Sie verbindet die Welten und stellt die Gleichheit in Position und Bewegung dar. Die Gravitation soll in späteren Leveln ebenfalls eine bedeutendere Rolle einnehmen, wobei Rätsel immer wieder mit ihr gelöst werden müssen. So können große Gegenstände nur durch Gravitation zwischen den Welten  ausgetauscht werden. Die Gravitation kann durch die Gravitationsmaschinen manipuliert werden, die nur eine geringe Reichweite haben und nicht lange an bleiben. Danach muss sich die Batterie erholen. Auch in Gegenden, in denen keine Rätsel vorhanden sind, soll die Maschine einen visuellen Effekt auslösen, damit der Spieler durch Ausprobieren erkennt, was die Maschinen bewirken.

In den folgenden Kapiteln sollen die für die Fortbewegung und für die Steuerung der Welt wichtigen Mechaniken beschrieben werden.