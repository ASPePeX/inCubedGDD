\mysubsection{Sarah Häfele}{Spielfluss}

Am Anfang des Spiels wird der Spieler durch einen Teleporter von der Realität auf die virtuelle Insel transportiert, die es zu Retten gilt. Der Spieler startet in Mitten des Würfels und kann sich dort in Ruhe umschauen. Durch das Anschauen von Viewpoints kann der Spieler alle Welten besuchen. Auf jeder Welt gilt es Rätsel zu lösen, die ebenfalls mit dem Blick ausgelöst werden. Werden alle Rätsel zweier gegenüberliegender Welten gelöst, erhält der Spieler ein Teil des zerbrochenen Remote Cubes - die Fernbedienung, die die Insel am Ende wieder aufklappen lässt.
Ziel des Spiels ist es, alle Welten zu besuchen, alle Rätsel zu lösen und die (im Prototyp) drei Teile der Fernbedienung zu sammeln. 
Der Spieler kann nicht verlieren. Am Ende sieht der Spieler, wie die Insel durch ihn gerettet wird und er kommt wieder zurück in die Realität.

Der Spielfluss kann beliebig ausgedehnt werden, indem neue Rätsel hinzukommen. Angedacht ist, dass alle Welten voneinander abhängig sind. Dies wird im Prototyp nur exemplarisch durch jeweils die gegenüberliegenden Welten umgesetzt.
