\subsection{Steuerung}
\mysubsubsection{Fabian Gärtner}{Physikalische Steuerung}

Da die Steuerung für den Spieler so einfach wie möglich gehalten werden soll, sodass auch Personen, die bisher nicht mit Videospielen und der entsprechenden Peripherie (also bspw. Controller) vertraut sind, inCubed spielen können, ist jegliche Interaktion mit dem Spiel durch Blickkontakt mit Interaktionspunkten möglich. Es ist allerdings, sei es aus Platzmangel oder aus gesundheitlichen Gründen, nicht immer möglich, sich vollständig nach hinten, oben oder unten umzusehen, was die Erkundung der Welt einschränken und damit das Lösen der Rätsel erschweren würde. Die Verwendung eines handelsüblichen Controllers zur Bewegung der Spielwelt wäre hier aber nicht sinnvoll, da dieser mit seiner Vielzahl an Tasten vor allem von ungeübten Spielern eher unintuitiv zu bedienen wäre, vor allem da durch Verwendung der VR-Brille der Controller nicht sichtbar ist. Daher steht für inCubed ein deutlich intuitiverer und unkonventioneller Controller in Form eines Würfels zur Verfügung, der sich sowohl in die Rahmenhandlung einfügt, als auch dem Spieler auf einfache Art und Weise ermöglicht, sich in der Welt umzusehen. Dazu muss er während des Spiels lediglich diesen Würfel vor die Kamera des Smartphones, das sich zwar in der VR-Brille befindet, aber durch das Sichtfenster in der VR-One die Umgebung filmen kann, halten und bei Bedarf so drehen, dass eine der sechs Würfelseiten zur Kamera zeigt. Dies führt gleichzeitig dazu, dass sich die Spielwelt entsprechend um jeweils 90° dreht und so auch andere Teile der Welt erkundet werden können, ohne dass sich der Spieler selbst umdrehen muss.

Das Design dieses Würfels, auf dessen sechs Seiten sich sechs unterschiedliche Formen (Dreieck, Stern, Rechteck, Kreis, Pentagon und Heptagon) befinden, wurde so gewählt, dass zum einen die Bewegung des Würfels durch die Kamera des Smartphones möglichst effizient und zuverlässig erkannt werden kann (mehr zum technischen Hintergrund in Kapitel X) und zum anderen so, dass er optimal zur Thematik von inCubed passt. Die Geschichte in inCubed erklärt, dass es sich bei diesem Würfel um den Hilferuf des verrückten Wissenschaftlers handelt. Da der Wissenschaftler von geometrischen Formen besessen scheint, hat er nicht nur den "Remote Cube" sondern auch diese SOS-Maschine in Form eines Würfels gebaut und mit simplen mathematischen Strukturen versehen. Da im Spiel gegenüberliegende Welten konträr sind, sind auch diese Formen auf gegenüberliegenden Seiten des Würfels so unterschiedlich wie möglich (bspw. das Dreieck mit seinen drei Eckpunkten und der Kreis, der technisch gesehen aus einer Vielzahl an Eckpunkten besteht). Sie haben aber dennoch die gleiche Farbe, da sie wie die gegenüberliegenden Welten im Spiel zusammengehören. Besonders am Ende des Spiels, nach Absetzen der Brille, wird dem Spieler dann auch bewusst, dass sein Controller starke Ähnlichkeit mit der Maschine hat, die durch das Lösen der Rätsel wieder zusammengesetzt werden musste.

Markerkennung

\mysubsubsection{Alexander Scheurer}{Virtuelle Steuerung}

Wenn Blicke steuern können.

