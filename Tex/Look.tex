\mysubsection{Lydia Friedrich}{Look und Atmosphäre}

Das Spiel ist im Low Poly Design umgesetzt, dafür gibt es drei Gründe:
\begin{itemize}
\item Low Poly ermöglicht eine schnelle Realisierung von komplexen 3D-Modellen innerhalb kurzer Zeit.
\item Weiterhin erzeugt Low Poly alternative stereoskopische Effekte und verstärkt dadurch den 3D-Effekt.
\item Auf Grund der reduzierten Polygonanzahl der jeweils einzelnen Modelle kann eine bessere Darstellbarkeit sowie Lauffähigkeit des Spiels realisiert werden.
\end{itemize}

Auf Grund des abstrakten Designs fühlt sich der Spieler in eine andere Realität, wie zum Beispiel einen Traum, versetzt. Die Farbgebung der Welt betont die Gegensätzlichkeit der Welten des Spiels, somit wird die Wahrnehmung des Spielers in der außergewöhnlichen Anordnung der einzelnen Objekte in den Welten unterstützt. Durch den freundlichen Gesamteindruck des Würfelinneren, entstehen keine negative Emotionen bei dem Spielenden, so dass sich dieser während des Spiels wohl fühlt.
