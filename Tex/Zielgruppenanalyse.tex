\mysubsection{Lydia Friedrich}{Zielgruppe}

Zur Zielgruppe des Spiels “InCubed” zählen Menschen ganz unterschiedlicher Charaktere. So können sich einerseits Menschen mit einer hohen Affinitiät im Bereich Technik und andererseits Casual Gamers für das Spiel interessieren. Auf Grund des Speildesigns, im Low Poly Stil, ist das Spiel ebenfalls für Kinder und Jugendliche geeignet. Augenmerk liegt jedoch eher auf Menschen, welche ein großes Interesse an Games mit neu entwickelten Techniken aufweisen und auch bereit sind für diese Produkte mehr Geld auszugeben als für alltägliche Games. Notwendig für eine korrekte Spielwiedergabe ist eine gute Hardware auf einem mobilen Endgerät sowie die VR-Brille von der Firma Zeiss, welche in der Anschaffung recht teuer sein können. Der Kauf einer VR-Brille kann jedoch für viele Casual Gamer ohne ein gewisses Maß an bereitgestellten Content unattraktiv sein. Die Tendenz von Spielumgebungen im mobilen Bereich ist jedoch steigend.

Da die Dialoge im Spiel momentan nur in Deutsch zur Verfügung stehen, ist das Spiel für den deutschen Markt und darüber hinaus für Menschen, welche der Deutschen Sprache mächtig sind, geeignet. Auch auf Grund des Rätselcharakters im Spiel, ist es für viele Personen im deutschsprachigen Raum interessant. Laut einer Studie aus dem Jahr 2014 - 2015, ist das Lösen von Rätseln auf der Beliebtheitsskala der Deutschen auf Platz drei (\footnote{\textit{Quelle:} \url{http://de.statista.com/statistik/daten/studie/171168/umfrage/haeufig-betriebene-freizeitaktivitaeten/}})
