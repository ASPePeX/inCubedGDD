\mysubsection{Lydia Friedrich}{Aufgaben}
\subsubsection{Gebirgswelt- Dorfwelt}
Der Spieler befindet sich im inneren des Würfels in welchem es stockfinster ist. Grund hierfür ist die zusammengefaltete Welt, in welcher sich der Spieler befindet (die Sonne kommt nicht mehr hindurch). Zudem ist die Lichtmaschine defekt, welche alle Welten des Würfels mit Strom versorgt. Um die Lichtmaschine einzuschalten und das erste Remote Cube Teil zu erhalten, muss der Spieler in der Gebirgswelt den Triggerpunkt der Gravitationsmaschine aktivieren, damit die Kristalle, welche sich am Gebirge befinden auf die Dorfwelt fallen. In der Dorfwelt muss ein Kristall nun in seine ursprüngliche Fassung hineingelegt werden. Um dies zu erreichen muss der Spieler den Kristall eine gewisse Zeit fixieren um dessen Triggerpunkt auszulösen. Nachdem sich der Kristall wieder in seiner Fassung befindet, erwacht der Würfel zu neuem Leben und es wird hell. Der Spieler kann sich nun alle Welten des Würfels genauer ansehen. Doch bevor der Spieler die Aufgaben der anderen Welten lösen kann, muss er sich zunächst das Remote Cube Teil bei dem winkenden Einwohner des Dorfes zum Dank abholen, in dem er den Triggerpunkt des Einwohners aktiviert und gleichzeitig wird der \enquote{Dankesdialog} aktiviert.


Diese erste Welt dient als eine Art Tutorial für den Spieler und er erhält somit die Möglichkeit sich mit der Spielumgebung und der Spielsteuerung vertraut zu machen.

\subsubsection{Eiswelt - Feuerwelt}
Um in diesem Weltenpaar das ersehnte Remote Cube Teil zu bekommen muss der Spieler zuerst auf der Feuerwelt den Vulkan ein gewissen Zeitraum ansehen, damit dessen Triggerpunkt ausgelöst wird und der Vulkan zu Spucken anfängt. Der Vulkan brodelt so stark, das vereinzelt Lavabrocken in die umliegende Landschaft fliegen und dabei das Haus des einzigen Bewohners der Welt in Brand setzen. Um dem Einwohner zu helfen, indem der Spieler das Feuer löscht, muss der Triggerpunkt des Einwohners ausgelöst werden, woraufhin der Spieler eine Fackel erhält. Mit dieser Fackel kann der Spieler in der Eiswelt den größten Eiszapfen zum Schmelzen bringen, indem er diesen lange fixiert und dadurch den Triggerpunkt des Eiszapfen auslöst. Im Anschluß muss der Spieler nur noch die Gravitationsmaschine durch einen Triggerpunkt aktivieren, damit das nun zu Wasser gewordene Eis auf das brennende Haus \enquote{hoch tropfen} kann und somit den Brand löscht. Der Spieler kann sich sein Remote Cube Teil beim Einwohner, mit Hilfe des Triggerpunktes, als Dank für seine Hilfe abholen.

\subsubsection{Wüstenwelt - Waldwelt}
Das Remote Cube Teil befindet sich in einem tiefen Loch in der Wüstenwelt. Eigenltich muss der Spieler nur das Gravitationsgerät einschalten, damit das Remote Cube Teil aus dem Loch heraus auf die Waldwelt fallen kann, jedoch ist das Gravitationsgerät unter Sand vergraben. Deshalb muss der Spieler zunächst in der Waldwelt eine Schaufel finden, welche an einen Holzstapel angelehnt ist. Durch Auslösen des Triggerspunkt, welcher sich auf der Schaufel befindet, kann der Spieler die Schaufel einsammeln. Mit Hilfe der Schaufel befreit der Spieler nun das Gravitationsgerät in der Wüstenwelt von dem Sand, dazu muss der Spieler auf den Sandhaufen über dem Gravitationsgerät den Triggerpunkt auslösen. Der Triggerpunkt auf dem Gravitationsgerät löst nun dessen Aktivität aus und das Remote Cube Teil fällt auf die Waldwelt und kann dort vom Spieler eingesammelt werden.